\subsection{Modular Inverse}
Fermat's little theorem states that
$$a^p \equiv a \pmod{p}$$
where $a$ is an integer and $p$ is a prime number.

The formula can be also represented as
$$a^{p - 1} \equiv 1 \pmod{p}$$

Normally, we call a number $b$ as an inverse of $a$ when $a \times b = 1$.

e.g. $0.5$ is the inverse of $2$.
\linebreak
\linebreak
Similarly, we call a number $b$ as the modular inverse of $a$ modulo $n$ when $a \times b \equiv 1 \pmod{n}$

e.g. 4 is the inverse of 2 modulo 7. ($4 \times 2 \bmod{7} = 1$)
\linebreak
\linebreak
If we recall the above formula, we can come into a conclusion that
the modular inverse of $a \bmod{p}$ is $a^{p - 2} \bmod{p}$ (for prime number $p$ only).
\linebreak
\linebreak
Code example:
\begin{verbatim}
// use the above bigmd function
long long modinv(long long a, long long p){
    return bigmd(a, p - 2, p);
}
\end{verbatim}