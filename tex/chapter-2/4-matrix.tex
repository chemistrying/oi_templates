\subsection{Matrix}
Matrices can be useful in solving linear recurrences.
\begin{lstlisting}
struct matrix {
    int row_size, col_size;
    vector<vector<long long>> container;
    matrix(int n, int m) : row_size(n), col_size(m), container(n, vector<long long>(m)) {};
    matrix(vector<vector<long long>> mtx) : row_size(mtx.size()), col_size(mtx.size() == 0 ? 0 : mtx[0].size()), container(mtx) {};
    
    matrix& operator *= (const matrix& rhs) {
        if (col_size != rhs.row_size) {
            cerr << "This two matrices cannot be multiplied together.\n";
            exit(EXIT_FAILURE);
        }
        
        vector<vector<long long>> new_container(row_size, vector<long long>(rhs.col_size));
        int common_side = col_size;
        for (int i = 0; i < row_size; i++) {
            for (int j = 0; j < rhs.col_size; j++) {
                long long result = 0;
                for (int k = 0; k < common_side; k++) {
                    result += container[i][k] * rhs.container[k][j];
                }
                new_container[i][j] = result;
            }
        }
    
        container = new_container;
        return *this;
    }
    
    friend matrix operator * (matrix lhs, const matrix& rhs) {
        lhs *= rhs;
        return lhs;
    }

    matrix& operator += (const matrix& rhs) {
        if (col_size != rhs.col_size || row_size != rhs.row_size) {
            cerr << "This two matrices cannot be added together.\n";
            exit(EXIT_FAILURE);
        }

        for (int i = 0; i < row_size; i++) {
            for (int j = 0; j < col_size; i++) {
                container[i][j] += rhs.container[i][j];
            }
        }

        return *this;
    }

    matrix operator + (matrix lhs, const matrix& rhs) {
        return lhs + rhs;
    }
    
    matrix& operator %= (const int mod) {
        for (int i = 0; i < row_size; i++) {
            for (int j = 0; j < col_size; j++) {
                container[i][j] %= mod;
            }
        }
        return *this;
    }
    
    friend matrix operator % (matrix lhs, const int mod) {
        lhs %= mod;
        return lhs;
    }
};
    
matrix power(matrix n, long long p, long long mod) {
    matrix curr = n;
    matrix result(vector<vector<long long>>({{1, 0}, {0, 1}}));
    while (p > 0) {
        if (p & 1) {
            result *= curr;
            result %= mod;
        }
        p >>= 1;
        curr *= curr;
        curr %= mod;
    }
    return result;
}
\end{lstlisting}