\subsection{Polynomial Operations}
Here, you can find polynomial operations.
\begin{lstlisting}
/* Verified on https://judge.yosupo.jp:
- N = 500'000:
-- Convolution, 440ms (https://judge.yosupo.jp/submission/85695)
-- Convolution (mod 1e9+7), 430ms (https://judge.yosupo.jp/submission/85696)
-- Inv of power series, 713ms (https://judge.yosupo.jp/submission/85694)
-- Exp of power series, 2157ms (https://judge.yosupo.jp/submission/85698)
-- Log of power series, 1181ms (https://judge.yosupo.jp/submission/85699)
-- Pow of power series, 3275ms (https://judge.yosupo.jp/submission/85703)
-- Sqrt of power series, 1919ms (https://judge.yosupo.jp/submission/85705)
-- P(x) -> P(x+a), 523ms (https://judge.yosupo.jp/submission/85706)
-- Division of polynomials, 996ms (https://judge.yosupo.jp/submission/85707)
- N = 100'000:
-- Multipoint evaluation, 2161ms (https://judge.yosupo.jp/submission/85709)
-- Polynomial interpolation, 2551ms (https://judge.yosupo.jp/submission/85711)
-- Kth term of Linear Recurrence, 2913ms (https://judge.yosupo.jp/submission/85727)
- N = 50'000:
-- Inv of Polynomials, 1691ms (https://judge.yosupo.jp/submission/85713)
- N = 10'000:
-- Find Linear Recurrence, 346ms (https://judge.yosupo.jp/submission/85025)
/////////

The main goal of this library is to implement common polynomial functionality in a
reasonable from competitive programming POV complexity, while also doing it in as
straight-forward way as possible.

Therefore, primary purpose of the library is educational and most of constant-time
optimizations that may significantly harm the code readability were not used.

The library is reasonably fast and generally can be used in most problems where
polynomial operations constitute intended solution. However, it is recommended to
seek out other implementations when the time limit is tight or you really want to
squeeze a solution when it is probably not the intended one.
*/

#include <bits/stdc++.h>

using namespace std;

namespace algebra {
    const int maxn = 1 << 20;
    const int magic = 250; // threshold for sizes to run the naive algo
    mt19937 rng(chrono::steady_clock::now().time_since_epoch().count()); 

    template<typename T>
    T bpow(T x, int64_t n) {
        if(n == 0) {
            return T(1);
        } else {
            auto t = bpow(x, n / 2);
            t = t * t;
            return n % 2 ? x * t : t;
        }
    }

    template<int m>
    struct modular {
        // https://en.wikipedia.org/wiki/Berlekamp-Rabin_algorithm
        // solves x^2 = y (mod m) assuming m is prime in O(log m).
        // returns nullopt if no sol.
        optional<modular> sqrt() const {
            static modular y;
            y = *this;
            if(r == 0) {
                return 0;
            } else if(bpow(y, (m - 1) / 2) != modular(1)) {
                return nullopt;
            } else {
                while(true) {
                    modular z = rng();
                    if(z * z == *this) {
                        return z;
                    }
                    struct lin {
                        modular a, b;
                        lin(modular a, modular b): a(a), b(b) {}
                        lin(modular a): a(a), b(0) {}
                        lin operator * (const lin& t) {
                            return {
                                a * t.a + b * t.b * y,
                                a * t.b + b * t.a
                            };
                        }
                    } x(z, 1); // z + x
                    x = bpow(x, (m - 1) / 2);
                    if(x.b != modular(0)) {
                        return x.b.inv();
                    }
                }
            }
        }
        
        int r;
        constexpr modular(): r(0) {}
        constexpr modular(int64_t rr): r(rr % m) {if(r < 0) r += m;}
        modular inv() const {return bpow(*this, m - 2);}
        modular operator - () const {return r ? m - r : 0;}
        modular operator * (const modular &t) const {return (int64_t)r * t.r % m;}
        modular operator / (const modular &t) const {return *this * t.inv();}
        modular operator += (const modular &t) {r += t.r; if(r >= m) r -= m; return *this;}
        modular operator -= (const modular &t) {r -= t.r; if(r < 0) r += m; return *this;}
        modular operator + (const modular &t) const {return modular(*this) += t;}
        modular operator - (const modular &t) const {return modular(*this) -= t;}
        modular operator *= (const modular &t) {return *this = *this * t;}
        modular operator /= (const modular &t) {return *this = *this / t;}
        
        bool operator == (const modular &t) const {return r == t.r;}
        bool operator != (const modular &t) const {return r != t.r;}
        
        explicit operator int() const {return r;}
        int64_t rem() const {return 2 * r > m ? r - m : r;}
    };
    
    template<int T>
    istream& operator >> (istream &in, modular<T> &x) {
        return in >> x.r;
    }
    
    template<int T>
    ostream& operator << (ostream &out, modular<T> const& x) {
        return out << x.r;
    }
    
    template<typename T>
    T fact(int n) {
        static T F[maxn];
        static bool init = false;
        if(!init) {
            F[0] = T(1);
            for(int i = 1; i < maxn; i++) {
                F[i] = F[i - 1] * T(i);
            }
            init = true;
        }
        return F[n];
    }
    
    template<typename T>
    T rfact(int n) {
        static T F[maxn];
        static bool init = false;
        if(!init) {
            F[maxn - 1] = T(1) / fact<T>(maxn - 1);
            for(int i = maxn - 2; i >= 0; i--) {
                F[i] = F[i + 1] * T(i + 1);
            }
            init = true;
        }
        return F[n];
    }

    template<typename T>
    T small_inv(int n) {
        static T F[maxn];
        static bool init = false;
        if(!init) {
            for(int i = 1; i < maxn; i++) {
                F[i] = rfact<T>(i) * fact<T>(i - 1);
            }
            init = true;
        }
        return F[n];
    }

    namespace fft {
        using ftype = double;
        struct point {
            ftype x, y;
            
            ftype real() {return x;}
            ftype imag() {return y;}
            
            point(): x(0), y(0){}
            point(ftype x, ftype y = 0): x(x), y(y){}
            
            static point polar(ftype rho, ftype ang) {
                return point{rho * cos(ang), rho * sin(ang)};
            }
            
            point conj() const {
                return {x, -y};
            }
            
            point operator +=(const point &t) {x += t.x, y += t.y; return *this;}
            point operator +(const point &t) const {return point(*this) += t;}
            point operator -(const point &t) const {return {x - t.x, y - t.y};}
            point operator *(const point &t) const {return {x * t.x - y * t.y, x * t.y + y * t.x};}
        };

        point w[maxn]; // w[2^n + k] = exp(pi * k / (2^n))
        int bitr[maxn];// b[2^n + k] = bitreverse(k)
        const ftype pi = acos(-1);
        bool initiated = 0;
        void init() {
            if(!initiated) {
                for(int i = 1; i < maxn; i *= 2) {
                    int ti = i / 2;
                    for(int j = 0; j < i; j++) {
                        w[i + j] = point::polar(ftype(1), pi * j / i);
                        if(ti) {
                            bitr[i + j] = 2 * bitr[ti + j % ti] + (j >= ti);
                        }
                    }
                }
                initiated = 1;
            }
        }
        
        void fft(auto &a, int n) {
            init();
            if(n == 1) {
                return;
            }
            int hn = n / 2;
            for(int i = 0; i < n; i++) {
                int ti = 2 * bitr[hn + i % hn] + (i > hn);
                if(i < ti) {
                    swap(a[i], a[ti]);
                }
            }
            for(int i = 1; i < n; i *= 2) {
                for(int j = 0; j < n; j += 2 * i) {
                    for(int k = j; k < j + i; k++) {
                        point t = a[k + i] * w[i + k - j];
                        a[k + i] = a[k] - t;
                        a[k] += t;
                    }
                }
            }
        }
        
        void mul_slow(vector<auto> &a, const vector<auto> &b) {
            if(a.empty() || b.empty()) {
                a.clear();
            } else {
                int n = a.size();
                int m = b.size();
                a.resize(n + m - 1);
                for(int k = n + m - 2; k >= 0; k--) {
                    a[k] *= b[0];
                    for(int j = max(k - n + 1, 1); j < min(k + 1, m); j++) {
                        a[k] += a[k - j] * b[j];
                    }
                }
            }
        }
        
        template<int m>
        struct dft {
            static constexpr int split = 1 << 15;
            vector<point> A;
            
            dft(vector<modular<m>> const& a, size_t n): A(n) {
                for(size_t i = 0; i < min(n, a.size()); i++) {
                    A[i] = point(
                        a[i].rem() % split,
                        a[i].rem() / split
                    );
                }
                if(n) {
                    fft(A, n);
                }
            }
        
            auto operator * (dft const& B) {
                assert(A.size() == B.A.size());
                size_t n = A.size();
                if(!n) {
                    return vector<modular<m>>();
                }
                vector<point> C(n), D(n);
                for(size_t i = 0; i < n; i++) {
                    C[i] = A[i] * (B[i] + B[(n - i) % n].conj());
                    D[i] = A[i] * (B[i] - B[(n - i) % n].conj());
                }
                fft(C, n);
                fft(D, n);
                reverse(begin(C) + 1, end(C));
                reverse(begin(D) + 1, end(D));
                int t = 2 * n;
                vector<modular<m>> res(n);
                for(size_t i = 0; i < n; i++) {
                    modular<m> A0 = llround(C[i].real() / t);
                    modular<m> A1 = llround(C[i].imag() / t + D[i].imag() / t);
                    modular<m> A2 = llround(D[i].real() / t);
                    res[i] = A0 + A1 * split - A2 * split * split;
                }
                return res;
            }
            
            point& operator [](int i) {return A[i];}
            point operator [](int i) const {return A[i];}
        };
        
        size_t com_size(size_t as, size_t bs) {
            if(!as || !bs) {
                return 0;
            }
            size_t n = as + bs - 1;
            while(__builtin_popcount(n) != 1) {
                n++;
            }
            return n;
        }
        
        template<int m>
        void mul(vector<modular<m>> &a, vector<modular<m>> b) {
            if(min(a.size(), b.size()) < magic) {
                mul_slow(a, b);
                return;
            }
            auto n = com_size(a.size(), b.size());
            auto A = dft<m>(a, n);
            if(a == b) {
                a = A * A;
            } else {
                a = A * dft<m>(b, n);
            }
        }
    }

    template<typename T>
    struct poly {
        vector<T> a;
        
        void normalize() { // get rid of leading zeroes
            while(!a.empty() && a.back() == T(0)) {
                a.pop_back();
            }
        }
        
        poly(){}
        poly(T a0) : a{a0}{normalize();}
        poly(const vector<T> &t) : a(t){normalize();}
        
        poly operator -() const {
            auto t = *this;
            for(auto &it: t.a) {
                it = -it;
            }
            return t;
        }
        
        poly operator += (const poly &t) {
            a.resize(max(a.size(), t.a.size()));
            for(size_t i = 0; i < t.a.size(); i++) {
                a[i] += t.a[i];
            }
            normalize();
            return *this;
        }
        
        poly operator -= (const poly &t) {
            a.resize(max(a.size(), t.a.size()));
            for(size_t i = 0; i < t.a.size(); i++) {
                a[i] -= t.a[i];
            }
            normalize();
            return *this;
        }
        poly operator + (const poly &t) const {return poly(*this) += t;}
        poly operator - (const poly &t) const {return poly(*this) -= t;}
        
        poly mod_xk(size_t k) const { // get first k coefficients
            return vector<T>(begin(a), begin(a) + min(k, a.size()));
        }
        
        poly mul_xk(size_t k) const { // multiply by x^k
            auto res = a;
            res.insert(begin(res), k, 0);
            return res;
        }
        
        poly div_xk(size_t k) const { // drop first k coefficients
            return vector<T>(begin(a) + min(k, a.size()), end(a));
        }
        
        poly substr(size_t l, size_t r) const { // return mod_xk(r).div_xk(l)
            return vector<T>(
                begin(a) + min(l, a.size()),
                begin(a) + min(r, a.size())
            );
        }
        
        poly operator *= (const poly &t) {fft::mul(a, t.a); normalize(); return *this;}
        poly operator * (const poly &t) const {return poly(*this) *= t;}
        
        poly reverse(size_t n) const { // computes x^n A(x^{-1})
            auto res = a;
            res.resize(max(n, res.size()));
            return vector<T>(res.rbegin(), res.rbegin() + n);
        }
        
        poly reverse() const {
            return reverse(deg() + 1);
        }
        
        pair<poly, poly> divmod_slow(const poly &b) const { // when divisor or quotient is small
            vector<T> A(a);
            vector<T> res;
            T b_lead_inv = b.a.back().inv();
            while(A.size() >= b.a.size()) {
                res.push_back(A.back() * b_lead_inv);
                if(res.back() != T(0)) {
                    for(size_t i = 0; i < b.a.size(); i++) {
                        A[A.size() - i - 1] -= res.back() * b.a[b.a.size() - i - 1];
                    }
                }
                A.pop_back();
            }
            std::reverse(begin(res), end(res));
            return {res, A};
        }
        
        pair<poly, poly> divmod_hint(poly const& b, poly const& binv) const { // when inverse is known
            assert(!b.is_zero());
            if(deg() < b.deg()) {
                return {poly{0}, *this};
            }
            int d = deg() - b.deg();
            if(min(d, b.deg()) < magic) {
                return divmod_slow(b);
            }
            poly D = (reverse().mod_xk(d + 1) * binv.mod_xk(d + 1)).mod_xk(d + 1).reverse(d + 1);
            return {D, *this - D * b};
        }
        
        pair<poly, poly> divmod(const poly &b) const { // returns quotiend and remainder of a mod b
            assert(!b.is_zero());
            if(deg() < b.deg()) {
                return {poly{0}, *this};
            }
            int d = deg() - b.deg();
            if(min(d, b.deg()) < magic) {
                return divmod_slow(b);
            }
            poly D = (reverse().mod_xk(d + 1) * b.reverse().inv(d + 1)).mod_xk(d + 1).reverse(d + 1);
            return {D, *this - D * b};
        }
        
        // (ax+b) / (cx+d)
        struct transform {
            poly a, b, c, d;
            transform(poly a, poly b = T(1), poly c = T(1), poly d = T(0)): a(a), b(b), c(c), d(d){}
            
            transform operator *(transform const& t) {
                return {
                    a*t.a + b*t.c, a*t.b + b*t.d,
                    c*t.a + d*t.c, c*t.b + d*t.d
                };
            }
            
            transform adj() {
                return transform(d, -b, -c, a);
            }
            
            auto apply(poly A, poly B) {
                return make_pair(a * A + b * B, c * A + d * B);
            }
        };
        
        template<typename Q>
        static void concat(vector<Q> &a, vector<Q> const& b) {
            for(auto it: b) {
                a.push_back(it);
            }
        }
        
        // finds a transform that changes A/B to A'/B' such that
        // deg B' is at least 2 times less than deg A
        static pair<vector<poly>, transform> half_gcd(poly A, poly B) {
            assert(A.deg() >= B.deg());
            int m = (A.deg() + 1) / 2;
            if(B.deg() < m) {
                return {{}, {T(1), T(0), T(0), T(1)}};
            }
            auto [ar, Tr] = half_gcd(A.div_xk(m), B.div_xk(m));
            tie(A, B) = Tr.adj().apply(A, B);
            if(B.deg() < m) {
                return {ar, Tr};
            }
            auto [ai, R] = A.divmod(B);
            tie(A, B) = make_pair(B, R);
            int k = 2 * m - B.deg();
            auto [as, Ts] = half_gcd(A.div_xk(k), B.div_xk(k));
            concat(ar, {ai});
            concat(ar, as);
            return {ar, Tr * transform(ai) * Ts};
        }
        
        // return a transform that reduces A / B to gcd(A, B) / 0
        static pair<vector<poly>, transform> full_gcd(poly A, poly B) {
            vector<poly> ak;
            vector<transform> trs;
            while(!B.is_zero()) {
                if(2 * B.deg() > A.deg()) {
                    auto [a, Tr] = half_gcd(A, B);
                    concat(ak, a);
                    trs.push_back(Tr);
                    tie(A, B) = trs.back().adj().apply(A, B);
                } else {
                    auto [a, R] = A.divmod(B);
                    ak.push_back(a);
                    trs.emplace_back(a);
                    tie(A, B) = make_pair(B, R);
                }
            }
            trs.emplace_back(T(1), T(0), T(0), T(1));
            while(trs.size() >= 2) {
                trs[trs.size() - 2] = trs[trs.size() - 2] * trs[trs.size() - 1];
                trs.pop_back();
            }
            return {ak, trs.back()};
        }
        
        static poly gcd(poly A, poly B) {
            if(A.deg() < B.deg()) {
                return full_gcd(B, A);
            }
            auto Tr = fraction(A, B);
            return Tr.d * A - Tr.b * B;
        }
        
        // Returns the characteristic polynomial
        // of the minimum linear recurrence for the sequence
        poly min_rec_slow(int d) const {
            auto R1 = mod_xk(d + 1).reverse(d + 1), R2 = xk(d + 1);
            auto Q1 = poly(T(1)), Q2 = poly(T(0));
            while(!R2.is_zero()) {
                auto [a, nR] = R1.divmod(R2); // R1 = a*R2 + nR, deg nR < deg R2
                tie(R1, R2) = make_tuple(R2, nR);
                tie(Q1, Q2) = make_tuple(Q2, Q1 + a * Q2);
                if(R2.deg() < Q2.deg()) {
                    return Q2 / Q2.lead();
                }
            }
            assert(0);
        }
        
        static transform convergent(auto L, auto R) { // computes product on [L, R)
            if(R - L == 1) {
                return transform(*L);
            } else {
                int s = 0;
                for(int i = 0; i < R - L; i++) {
                    s += L[i].a.size();
                }
                int c = 0;
                for(int i = 0; i < R - L; i++) {
                    c += L[i].a.size();
                    if(2 * c > s) {
                        return convergent(L, L + i) * convergent(L + i, R);
                    }
                }
                assert(0);
            }
        }
        
        poly min_rec(int d) const {
            if(d < magic) {
                return min_rec_slow(d);
            }
            auto R2 = mod_xk(d + 1).reverse(d + 1), R1 = xk(d + 1);
            if(R2.is_zero()) {
                return poly(1);
            }
            auto [a, Tr] = full_gcd(R1, R2);
            int dr = (d + 1) - a[0].deg();
            int dp = 0;
            for(size_t i = 0; i + 1 < a.size(); i++) {
                dr -= a[i + 1].deg();
                dp += a[i].deg();
                if(dr < dp) {
                    auto ans = convergent(begin(a), begin(a) + i + 1);
                    return ans.a / ans.a.lead();
                }
            }
            auto ans = convergent(begin(a), end(a));
            return ans.a / ans.a.lead();
        }
        
        // calculate inv to *this modulo t
        // quadratic complexity
        optional<poly> inv_mod_slow(poly const& t) const {
            auto R1 = *this, R2 = t;
            auto Q1 = poly(T(1)), Q2 = poly(T(0));
            int k = 0;
            while(!R2.is_zero()) {
                k ^= 1;
                auto [a, nR] = R1.divmod(R2);
                tie(R1, R2) = make_tuple(R2, nR);
                tie(Q1, Q2) = make_tuple(Q2, Q1 + a * Q2);
            }
            if(R1.deg() > 0) {
                return nullopt;
            } else {
                return (k ? -Q1 : Q1) / R1[0];
            }
        }
        
        optional<poly> inv_mod(poly const &t) const {
            assert(!t.is_zero());
            if(false && min(deg(), t.deg()) < magic) {
                return inv_mod_slow(t);
            }
            auto A = t, B = *this % t;
            auto [a, Tr] = full_gcd(A, B);
            auto g = Tr.d * A - Tr.b * B;
            if(g.deg() != 0) {
                return nullopt;
            }
            return -Tr.b / g[0];
        };
        
        poly operator / (const poly &t) const {return divmod(t).first;}
        poly operator % (const poly &t) const {return divmod(t).second;}
        poly operator /= (const poly &t) {return *this = divmod(t).first;}
        poly operator %= (const poly &t) {return *this = divmod(t).second;}
        poly operator *= (const T &x) {
            for(auto &it: a) {
                it *= x;
            }
            normalize();
            return *this;
        }
        poly operator /= (const T &x) {
            return *this *= x.inv();
        }
        poly operator * (const T &x) const {return poly(*this) *= x;}
        poly operator / (const T &x) const {return poly(*this) /= x;}
        
        poly conj() const { // A(x) -> A(-x)
            auto res = *this;
            for(int i = 1; i <= deg(); i += 2) {
                res.a[i] = -res[i];
            }
            return res;
        }
        
        void print(int n) const {
            for(int i = 0; i < n; i++) {
                cout << (*this)[i] << ' ';
            }
            cout << "\n";
        }
        
        void print() const {
            print(deg() + 1);
        }
        
        T eval(T x) const { // evaluates in single point x
            T res(0);
            for(int i = deg(); i >= 0; i--) {
                res *= x;
                res += a[i];
            }
            return res;
        }
        
        T lead() const { // leading coefficient
            assert(!is_zero());
            return a.back();
        }
        
        int deg() const { // degree, -1 for P(x) = 0
            return (int)a.size() - 1;
        }
        
        bool is_zero() const {
            return a.empty();
        }
        
        T operator [](int idx) const {
            return idx < 0 || idx > deg() ? T(0) : a[idx];
        }
        
        T& coef(size_t idx) { // mutable reference at coefficient
            return a[idx];
        }
        
        bool operator == (const poly &t) const {return a == t.a;}
        bool operator != (const poly &t) const {return a != t.a;}
        
        poly deriv(int k = 1) { // calculate derivative
            if(deg() + 1 < k) {
                return poly(T(0));
            }
            vector<T> res(deg() + 1 - k);
            for(int i = k; i <= deg(); i++) {
                res[i - k] = fact<T>(i) * rfact<T>(i - k) * a[i];
            }
            return res;
        }
        
        poly integr() { // calculate integral with C = 0
            vector<T> res(deg() + 2);
            for(int i = 0; i <= deg(); i++) {
                res[i + 1] = a[i] * small_inv<T>(i + 1);
            }
            return res;
        }
        
        size_t trailing_xk() const { // Let p(x) = x^k * t(x), return k
            if(is_zero()) {
                return -1;
            }
            int res = 0;
            while(a[res] == T(0)) {
                res++;
            }
            return res;
        }
        
        poly log(size_t n) { // calculate log p(x) mod x^n
            assert(a[0] == T(1));
            return (deriv().mod_xk(n) * inv(n)).integr().mod_xk(n);
        }
        
        poly exp(size_t n) { // calculate exp p(x) mod x^n
            if(is_zero()) {
                return T(1);
            }
            assert(a[0] == T(0));
            poly ans = T(1);
            size_t a = 1;
            while(a < n) {
                poly C = ans.log(2 * a).div_xk(a) - substr(a, 2 * a);
                ans -= (ans * C).mod_xk(a).mul_xk(a);
                a *= 2;
            }
            return ans.mod_xk(n);
        }
        
        poly pow_bin(int64_t k, size_t n) { // O(n log n log k)
            if(k == 0) {
                return poly(1).mod_xk(n);
            } else {
                auto t = pow(k / 2, n);
                t = (t * t).mod_xk(n);
                return (k % 2 ? *this * t : t).mod_xk(n);
            }
        }
        
        // Do not compute inverse from scratch
        poly powmod_hint(int64_t k, poly const& md, poly const& mdinv) {
            if(k == 0) {
                return poly(1);
            } else {
                auto t = powmod_hint(k / 2, md, mdinv);
                t = (t * t).divmod_hint(md, mdinv).second;
                if(k % 2) {
                    t = (t * *this).divmod_hint(md, mdinv).second;
                }
                return t;
            }
        }

        poly circular_closure(size_t m) const {
            if(deg() == -1) {
                return *this;
            }
            auto t = *this;
            for(size_t i = t.deg(); i >= m; i--) {
                t.a[i - m] += t.a[i];
            }
            t.a.resize(min(t.a.size(), m));
            return t;
        }

        static poly mul_circular(poly const& a, poly const& b, size_t m) {
            return (a.circular_closure(m) * b.circular_closure(m)).circular_closure(m);
        }

        poly powmod_circular(int64_t k, size_t m) {
            if(k == 0) {
                return poly(1);
            } else {
                auto t = powmod_circular(k / 2, m);
                t = mul_circular(t, t, m);
                if(k % 2) {
                    t = mul_circular(t, *this, m);
                }
                return t;
            }
        }
        
        poly powmod(int64_t k, poly const& md) {
            int d = md.deg();
            if(d == -1) {
                return k ? *this : poly(T(1));
            }
            if(md == xk(d)) {
                return pow(k, d);
            }
            if(md == xk(d) - poly(T(1))) {
                return powmod_circular(k, d);
            }
            auto mdinv = md.reverse().inv(md.deg() + 1);
            return powmod_hint(k, md, mdinv);
        }
        
        // O(d * n) with the derivative trick from
        // https://codeforces.com/blog/entry/73947?#comment-581173
        poly pow_dn(int64_t k, size_t n) {
            if(n == 0) {
                return poly(T(0));
            }
            assert((*this)[0] != T(0));
            vector<T> Q(n);
            Q[0] = bpow(a[0], k);
            auto a0inv = a[0].inv();
            for(int i = 1; i < (int)n; i++) {
                for(int j = 1; j <= min(deg(), i); j++) {
                    Q[i] += a[j] * Q[i - j] * (T(k) * T(j) - T(i - j));
                }
                Q[i] *= small_inv<T>(i) * a0inv;
            }
            return Q;
        }
        
        // calculate p^k(n) mod x^n in O(n log n)
        // might be quite slow due to high constant
        poly pow(int64_t k, size_t n) {
            if(is_zero()) {
                return k ? *this : poly(1);
            }
            int i = trailing_xk();
            if(i > 0) {
                return k >= int64_t(n + i - 1) / i ? poly(T(0)) : div_xk(i).pow(k, n - i * k).mul_xk(i * k);
            }
            if(min(deg(), (int)n) <= magic) {
                return pow_dn(k, n);
            }
            if(k <= magic) {
                return pow_bin(k, n);
            }
            T j = a[i];
            poly t = *this / j;
            return bpow(j, k) * (t.log(n) * T(k)).exp(n).mod_xk(n);
        }
        
        // returns nullopt if undefined
        optional<poly> sqrt(size_t n) const {
            if(is_zero()) {
                return *this;
            }
            int i = trailing_xk();
            if(i % 2) {
                return nullopt;
            } else if(i > 0) {
                auto ans = div_xk(i).sqrt(n - i / 2);
                return ans ? ans->mul_xk(i / 2) : ans;
            }
            auto st = (*this)[0].sqrt();
            if(st) {
                poly ans = *st;
                size_t a = 1;
                while(a < n) {
                    a *= 2;
                    ans -= (ans - mod_xk(a) * ans.inv(a)).mod_xk(a) / 2;
                }
                return ans.mod_xk(n);
            }
            return nullopt;
        }
        
        poly mulx(T a) const { // component-wise multiplication with a^k
            T cur = 1;
            poly res(*this);
            for(int i = 0; i <= deg(); i++) {
                res.coef(i) *= cur;
                cur *= a;
            }
            return res;
        }

        poly mulx_sq(T a) const { // component-wise multiplication with a^{k choose 2}
            T cur = 1, total = 1;
            poly res(*this);
            for(int i = 0; i <= deg(); i++) {
                res.coef(i) *= total;
                cur *= a;
                total *= cur;
            }
            return res;
        }

        // be mindful of maxn, as the function
        // requires multiplying polynomials of size deg() and n+deg()!
        poly chirpz(T z, int n) const { // P(1), P(z), P(z^2), ..., P(z^(n-1))
            if(is_zero()) {
                return vector<T>(n, 0);
            }
            if(z == T(0)) {
                vector<T> ans(n, (*this)[0]);
                if(n > 0) {
                    ans[0] = accumulate(begin(a), end(a), T(0));
                }
                return ans;
            }
            auto A = mulx_sq(z.inv());
            auto B = ones(n+deg()).mulx_sq(z);
            return semicorr(B, A).mod_xk(n).mulx_sq(z.inv());
        }

        // res[i] = prod_{1 <= j <= i} 1/(1 - z^j)
        static auto _1mzk_prod_inv(T z, int n) {
            vector<T> res(n, 1), zk(n);
            zk[0] = 1;
            for(int i = 1; i < n; i++) {
                zk[i] = zk[i - 1] * z;
                res[i] = res[i - 1] * (T(1) - zk[i]);
            }
            res.back() = res.back().inv();
            for(int i = n - 2; i >= 0; i--) {
                res[i] = (T(1) - zk[i+1]) * res[i+1];
            }
            return res;
        }
        
        // prod_{0 <= j < n} (1 - z^j x)
        static auto _1mzkx_prod(T z, int n) {
            if(n == 1) {
                return poly(vector<T>{1, -1});
            } else {
                auto t = _1mzkx_prod(z, n / 2);
                t *= t.mulx(bpow(z, n / 2));
                if(n % 2) {
                    t *= poly(vector<T>{1, -bpow(z, n - 1)});
                }
                return t;
            }
        }

        poly chirpz_inverse(T z, int n) const { // P(1), P(z), P(z^2), ..., P(z^(n-1))
            if(is_zero()) {
                return {};
            }
            if(z == T(0)) {
                if(n == 1) {
                    return *this;
                } else {
                    return vector{(*this)[1], (*this)[0] - (*this)[1]};
                }
            }
            vector<T> y(n);
            for(int i = 0; i < n; i++) {
                y[i] = (*this)[i];
            }
            auto prods_pos = _1mzk_prod_inv(z, n);
            auto prods_neg = _1mzk_prod_inv(z.inv(), n);

            T zn = bpow(z, n-1).inv();
            T znk = 1;
            for(int i = 0; i < n; i++) {
                y[i] *= znk * prods_neg[i] * prods_pos[(n - 1) - i];
                znk *= zn;
            }

            poly p_over_q = poly(y).chirpz(z, n);
            poly q = _1mzkx_prod(z, n);

            return (p_over_q * q).mod_xk(n).reverse(n);
        }

        static poly build(vector<poly> &res, int v, auto L, auto R) { // builds evaluation tree for (x-a1)(x-a2)...(x-an)
            if(R - L == 1) {
                return res[v] = vector<T>{-*L, 1};
            } else {
                auto M = L + (R - L) / 2;
                return res[v] = build(res, 2 * v, L, M) * build(res, 2 * v + 1, M, R);
            }
        }

        poly to_newton(vector<poly> &tree, int v, auto l, auto r) {
            if(r - l == 1) {
                return *this;
            } else {
                auto m = l + (r - l) / 2;
                auto A = (*this % tree[2 * v]).to_newton(tree, 2 * v, l, m);
                auto B = (*this / tree[2 * v]).to_newton(tree, 2 * v + 1, m, r);
                return A + B.mul_xk(m - l);
            }
        }

        poly to_newton(vector<T> p) {
            if(is_zero()) {
                return *this;
            }
            int n = p.size();
            vector<poly> tree(4 * n);
            build(tree, 1, begin(p), end(p));
            return to_newton(tree, 1, begin(p), end(p));
        }

        vector<T> eval(vector<poly> &tree, int v, auto l, auto r) { // auxiliary evaluation function
            if(r - l == 1) {
                return {eval(*l)};
            } else {
                auto m = l + (r - l) / 2;
                auto A = (*this % tree[2 * v]).eval(tree, 2 * v, l, m);
                auto B = (*this % tree[2 * v + 1]).eval(tree, 2 * v + 1, m, r);
                A.insert(end(A), begin(B), end(B));
                return A;
            }
        }
        
        vector<T> eval(vector<T> x) { // evaluate polynomial in (x1, ..., xn)
            int n = x.size();
            if(is_zero()) {
                return vector<T>(n, T(0));
            }
            vector<poly> tree(4 * n);
            build(tree, 1, begin(x), end(x));
            return eval(tree, 1, begin(x), end(x));
        }
        
        poly inter(vector<poly> &tree, int v, auto l, auto r, auto ly, auto ry) { // auxiliary interpolation function
            if(r - l == 1) {
                return {*ly / a[0]};
            } else {
                auto m = l + (r - l) / 2;
                auto my = ly + (ry - ly) / 2;
                auto A = (*this % tree[2 * v]).inter(tree, 2 * v, l, m, ly, my);
                auto B = (*this % tree[2 * v + 1]).inter(tree, 2 * v + 1, m, r, my, ry);
                return A * tree[2 * v + 1] + B * tree[2 * v];
            }
        }
        
        static auto inter(vector<T> x, vector<T> y) { // interpolates minimum polynomial from (xi, yi) pairs
            int n = x.size();
            vector<poly> tree(4 * n);
            return build(tree, 1, begin(x), end(x)).deriv().inter(tree, 1, begin(x), end(x), begin(y), end(y));
        }

        static auto resultant(poly a, poly b) { // computes resultant of a and b
            if(b.is_zero()) {
                return 0;
            } else if(b.deg() == 0) {
                return bpow(b.lead(), a.deg());
            } else {
                int pw = a.deg();
                a %= b;
                pw -= a.deg();
                auto mul = bpow(b.lead(), pw) * T((b.deg() & a.deg() & 1) ? -1 : 1);
                auto ans = resultant(b, a);
                return ans * mul;
            }
        }
                
        static poly xk(size_t n) { // P(x) = x^n
            return poly(T(1)).mul_xk(n);
        }
        
        static poly ones(size_t n) { // P(x) = 1 + x + ... + x^{n-1} 
            return vector<T>(n, 1);
        }
        
        static poly expx(size_t n) { // P(x) = e^x (mod x^n)
            return ones(n).borel();
        }

        static poly log1px(size_t n) { // P(x) = log(1+x) (mod x^n)
            vector<T> coeffs(n, 0);
            for(size_t i = 1; i < n; i++) {
                coeffs[i] = (i & 1 ? T(i).inv() : -T(i).inv());
            }
            return coeffs;
        }

        static poly log1mx(size_t n) { // P(x) = log(1-x) (mod x^n)
            return -ones(n).integr();
        }
        
        // [x^k] (a corr b) = sum_{i} a{(k-m)+i}*bi
        static poly corr(poly a, poly b) { // cross-correlation
            return a * b.reverse();
        }

        // [x^k] (a semicorr b) = sum_i a{i+k} * b{i}
        static poly semicorr(poly a, poly b) {
            return corr(a, b).div_xk(b.deg());
        }
        
        poly invborel() const { // ak *= k!
            auto res = *this;
            for(int i = 0; i <= deg(); i++) {
                res.coef(i) *= fact<T>(i);
            }
            return res;
        }
        
        poly borel() const { // ak /= k!
            auto res = *this;
            for(int i = 0; i <= deg(); i++) {
                res.coef(i) *= rfact<T>(i);
            }
            return res;
        }
        
        poly shift(T a) const { // P(x + a)
            return semicorr(invborel(), expx(deg() + 1).mulx(a)).borel();
        }
        
        poly x2() { // P(x) -> P(x^2)
            vector<T> res(2 * a.size());
            for(size_t i = 0; i < a.size(); i++) {
                res[2 * i] = a[i];
            }
            return res;
        }
        
        // Return {P0, P1}, where P(x) = P0(x) + xP1(x)
        pair<poly, poly> bisect() const {
            vector<T> res[2];
            res[0].reserve(deg() / 2 + 1);
            res[1].reserve(deg() / 2 + 1);
            for(int i = 0; i <= deg(); i++) {
                res[i % 2].push_back(a[i]);
            }
            return {res[0], res[1]};
        }
        
        // Find [x^k] P / Q
        static T kth_rec(poly P, poly Q, int64_t k) {
            while(k > Q.deg()) {
                int n = Q.a.size();
                auto [Q0, Q1] = Q.mulx(-1).bisect();
                auto [P0, P1] = P.bisect();
                
                int N = fft::com_size((n + 1) / 2, (n + 1) / 2);
                
                auto Q0f = fft::dft(Q0.a, N);
                auto Q1f = fft::dft(Q1.a, N);
                auto P0f = fft::dft(P0.a, N);
                auto P1f = fft::dft(P1.a, N);
                
                if(k % 2) {
                    P = poly(Q0f * P1f) + poly(Q1f * P0f);
                } else {
                    P = poly(Q0f * P0f) + poly(Q1f * P1f).mul_xk(1);
                }
                Q = poly(Q0f * Q0f) - poly(Q1f * Q1f).mul_xk(1);
                k /= 2;
            }
            return (P * Q.inv(Q.deg() + 1))[k];
        }
        
        poly inv(int n) const { // get inverse series mod x^n
            auto Q = mod_xk(n);
            if(n == 1) {
                return Q[0].inv();
            }
            // Q(-x) = P0(x^2) + xP1(x^2)
            auto [P0, P1] = Q.mulx(-1).bisect();
            
            int N = fft::com_size((n + 1) / 2, (n + 1) / 2);
            
            auto P0f = fft::dft(P0.a, N);
            auto P1f = fft::dft(P1.a, N);
            
            auto TTf = fft::dft(( // Q(x)*Q(-x) = Q0(x^2)^2 - x^2 Q1(x^2)^2
                poly(P0f * P0f) - poly(P1f * P1f).mul_xk(1)
            ).inv((n + 1) / 2).a, N);
            
            return (
                poly(P0f * TTf).x2() + poly(P1f * TTf).x2().mul_xk(1)
            ).mod_xk(n);
        }
        
        // compute A(B(x)) mod x^n in O(n^2)
        static poly compose(poly A, poly B, int n) {
            int q = std::sqrt(n);
            vector<poly> Bk(q);
            auto Bq = B.pow(q, n);
            Bk[0] = poly(T(1));
            for(int i = 1; i < q; i++) {
                Bk[i] = (Bk[i - 1] * B).mod_xk(n);
            }
            poly Bqk(1);
            poly ans;
            for(int i = 0; i <= n / q; i++) {
                poly cur;
                for(int j = 0; j < q; j++) {
                    cur += Bk[j] * A[i * q + j];
                }
                ans += (Bqk * cur).mod_xk(n);
                Bqk = (Bqk * Bq).mod_xk(n);
            }
            return ans;
        }
        
        // compute A(B(x)) mod x^n in O(sqrt(pqn log^3 n))
        // preferrable when p = deg A and q = deg B
        // are much less than n
        static poly compose_large(poly A, poly B, int n) {
            if(B[0] != T(0)) {
                return compose_large(A.shift(B[0]), B - B[0], n);
            }
            
            int q = std::sqrt(n);
            auto [B0, B1] = make_pair(B.mod_xk(q), B.div_xk(q));
            
            B0 = B0.div_xk(1);
            vector<poly> pw(A.deg() + 1);
            auto getpow = [&](int k) {
                return pw[k].is_zero() ? pw[k] = B0.pow(k, n - k) : pw[k];
            };
            
            function<poly(poly const&, int, int)> compose_dac = [&getpow, &compose_dac](poly const& f, int m, int N) {
                if(f.deg() <= 0) {
                    return f;
                }
                int k = m / 2;
                auto [f0, f1] = make_pair(f.mod_xk(k), f.div_xk(k));
                auto [A, B] = make_pair(compose_dac(f0, k, N), compose_dac(f1, m - k, N - k));
                return (A + (B.mod_xk(N - k) * getpow(k).mod_xk(N - k)).mul_xk(k)).mod_xk(N);
            };
            
            int r = n / q;
            auto Ar = A.deriv(r);
            auto AB0 = compose_dac(Ar, Ar.deg() + 1, n);
            
            auto Bd = B0.mul_xk(1).deriv();
            
            poly ans = T(0);
            
            vector<poly> B1p(r + 1);
            B1p[0] = poly(T(1));
            for(int i = 1; i <= r; i++) {
                B1p[i] = (B1p[i - 1] * B1.mod_xk(n - i * q)).mod_xk(n - i * q);
            }
            while(r >= 0) {
                ans += (AB0.mod_xk(n - r * q) * rfact<T>(r) * B1p[r]).mul_xk(r * q).mod_xk(n);
                r--;
                if(r >= 0) {
                    AB0 = ((AB0 * Bd).integr() + A[r] * fact<T>(r)).mod_xk(n);
                }
            }
            
            return ans;
        }
    };
    
    static auto operator * (const auto& a, const poly<auto>& b) {
        return b * a;
    }
};

using namespace algebra;

const int mod = 998244353;
typedef modular<mod> base;
typedef poly<base> polyn;

void solve() {
    int n;
    cin >> n;
    vector<base> f(n), g(n);
    copy_n(istream_iterator<base>(cin), n, begin(f));
    copy_n(istream_iterator<base>(cin), n, begin(g));
    polyn::compose(f, g, n).print(n);
}

signed main() {
    //freopen("input.txt", "r", stdin);
    ios::sync_with_stdio(0);
    cin.tie(0);
    int t;
    t = 1;// cin >> t;
    while(t--) {
        solve();
    }
}
\end{lstlisting}