\subsection{Segment Tree}
\subsubsection{Basic Segment Tree}
\begin{lstlisting}
template <class T>
class segment_tree {
private:
    int n;
    vector<T> tree;
    T operation(T left, T right) {
        return left + right;
    }
 
    void pushup(int node) {
        tree[node] = operation(tree[node + node], tree[node + node + 1]);
    }
 
    void build(const vector<T>& v, int curr, int lo, int hi) {
        if (lo == hi) {
            tree[curr] = v[lo];
        } else {
            int mi = lo + (hi - lo) / 2;
            build(v, curr + curr, lo, mi);
            build(v, curr + curr + 1, mi + 1, hi);
            pushup(curr);
        }
    }
 
    void set(int idx, int val, int curr, int lo, int hi) {
        if (lo == hi) {
            tree[curr] = val;
        } else {
            int mi = lo + (hi - lo) / 2;
            if (idx <= mi) {
                set(idx, val, curr + curr, lo, mi);
            } else {
                set(idx, val, curr + curr + 1, mi + 1, hi);
            }
            pushup(curr);
        }
    }
 
    T query(int l, int r, int curr, int lo, int hi) { // l, r is the target range!
        if (lo == l && hi == r) {
            return tree[curr];
        } else {
            int mi = lo + (hi - lo) / 2;
            if (r <= mi) {
                return query(l, r, curr + curr, lo, mi);
            } else if (l >= mi + 1) {
                return query(l, r, curr + curr + 1, mi + 1, hi);
            } else {
                return operation(query(l, mi, curr + curr, lo, mi), query(mi + 1, r, curr + curr + 1, mi + 1, hi));
            }
        }
    }
 
public:
    segment_tree(const vector<T>& v) : n(v.size()), tree(4 * v.size()) {
        build(v, 1, 0, n - 1);
    }
 
    void set(int idx, T val) {
        set(idx, val, 1, 0, n - 1);
    }
 
    T query(int l, int r) {
        return query(l, r, 1, 0, n - 1);
    }
};
\end{lstlisting}

\subsubsection{Online Segment with Maximum Sum}
\begin{lstlisting}
class segment_tree {
    int _size = 0;
    int power2size = 1;
    vector<long long> v;
    vector<long long> pref, suf, sum;
 
    public:
        segment_tree(int length) {
            _size = length;
            v.resize(4 * _size, 0);
            pref.resize(4 * _size);
            suf.resize(4 * _size);
            sum.resize(4 * _size);
 
            length--;
            while (length > 0) power2size <<= 1, length >>= 1;
        }
 
        void set(int l, int r, int pos, int idx, long long x, int p2s) {
            if (inRange(pos, pos + 1, l, r) == 2) {
                int mid = p2s >> 1;
                if (r - l != 1) {
                    set(l, l + mid, pos, 2 * idx + 1, x, mid);
                    set(l + mid, r, pos, 2 * idx + 2, x, mid);
                    v[idx] = max(max(v[2 * idx + 1], v[2 * idx + 2]), suf[2 * idx + 1] + pref[2 * idx + 2]);
                    pref[idx] = max(pref[2 * idx + 1], sum[2 * idx + 1] + pref[2 * idx + 2]);
                    suf[idx] = max(suf[2 * idx + 2], sum[2 * idx + 2] + suf[2 * idx + 1]);
                    sum[idx] = sum[2 * idx + 1] + sum[2 * idx + 2];
                } else {
                    v[idx] = max(0LL, x);
                    pref[idx] = max(0LL, x);
                    suf[idx] = max(0LL, x);
                    sum[idx] = x;
                }
            }
        }
 
        void set(int pos, long long x) {
            set(0, _size, pos, 0, x, power2size);
        }
 
        // current left, current right, target left, target right
        long long ans(int l, int r, int lt, int rt, int idx, int p2s) {
            int checker = inRange(l, r, lt, rt);
            int mid = p2s >> 1;
            if (checker == 0) return 0;
            else if (checker == 1) return max(ans(l, l + mid, lt, rt, 2 * idx + 1, mid), ans(l + mid, r, lt, rt, 2 * idx + 2, mid));
            else return v[idx];
        }
 
        // [l, r)
        long long ans(int l, int r) {
            return ans(0, _size, l, r, 0, power2size);
        }
        
        int size() {
            return _size;
        }
        
        friend istream& operator >> (istream& is, segment_tree& obj) {
            for (int i = 0; i < obj.size(); i++) {
                int x;
                is >> x;
                obj.set(i, x);
            }
            return is;
        }
 
    private:
        // 0 = completely out of range, 1 = partially in range, 2 = completely in range
        // current segment, target segment
        int inRange(int l, int r, int lt, int rt) {
            if (r <= lt || l >= rt) return 0;
            else if (l >= lt && r <= rt) return 2;
            else return 1;
        }
};
\end{lstlisting}
\subsubsection{Lazy Segment Tree}
\begin{lstlisting}
template <class T>
class segment_tree {
private:
    int n;
    vector<T> tree, diff; // tree = real value, diff = modification value
    vector<bool> lazy; // lazy = is the current node lazied

    T operation(T left, T right) {
        return min(left, right); 
    }

    void pushup(int node) {
        if (lazy[node]) {
            // un-lazy itself
            lazy[node] = false;
            // lazy its children
            tree[node + node] = tree[node + node + 1] = diff[node];
            diff[node + node] = diff[node + node + 1] = diff[node];
            lazy[node + node] = lazy[node + node + 1] = true;
        }
        tree[node] = operation(tree[node + node], tree[node + node + 1]);
    }

    void build(const vector<T>& v, int curr, int lo, int hi) {
        if (lo == hi) {
            tree[curr] = v[lo];
        } else {
            int mi = lo + (hi - lo) / 2;
            build(v, curr + curr, lo, mi);
            build(v, curr + curr + 1, mi + 1, hi);
            pushup(curr);
        }
    }

    void set(int l, int r, T val, int curr, int lo, int hi) {
        if (lo == l && hi == r) {
            tree[curr] = val;
            diff[curr] = val;
            lazy[curr] = true;
        } else {
            pushup(curr);
            int mi = lo + (hi - lo) / 2;
            if (r <= mi) {
                set(l, r, val, curr + curr, lo, mi);
            } else if (l >= mi + 1) {
                set(l, r, val, curr + curr + 1, mi + 1, hi);
            } else {
                set(l, mi, val, curr + curr, lo, mi);
                set(mi + 1, r, val, curr + curr + 1, mi + 1, hi);
            }
            pushup(curr);
        }
    }

    T query(int l, int r, int curr, int lo, int hi) {
        if (lo == l && hi == r) {
            return tree[curr];
        } else {
            pushup(curr);
            int mi = lo + (hi - lo) / 2;
            if (r <= mi) {
                return query(l, r, curr + curr, lo, mi);
            } else if (l >= mi + 1) {
                return query(l, r, curr + curr + 1, mi + 1, hi);
            } else {
                return operation(query(l, mi, curr + curr, lo, mi), query(mi + 1, r, curr + curr + 1, mi + 1, hi));
            }
        }
    }
public:
    segment_tree(const vector<T>& v) : n(v.size()), tree(4 * v.size()), diff(4 * v.size()), lazy(4 * v.size()) {
        build(v, 1, 0, n - 1);
    }

    void set(int l, int r, T val) {
        set(l, r, val, 1, 0, n - 1);
        // cout << '\n';
    }

    T query(int l, int r) {
        return query(l, r, 1, 0, n - 1);
    }
};
\end{lstlisting}