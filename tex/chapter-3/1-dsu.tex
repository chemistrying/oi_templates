\subsection{Disjoint-set Union}
This data structure provides the following capabilities. 
We are given several elements, each of which is a separate set. 
A DSU will have an operation to combine any two sets, 
and it will be able to tell in which set a specific element is. 
The structure is very flexible as you can add different operations to it.
\begin{lstlisting}
class disjoint_set_union {
    vector<int> parent; // ancestor
    vector<int> minn, maxx;
    vector<int> count;
    
    vector<bool> config;
    public:
        disjoint_set_union(int length) {
            parent.resize(length + 1);
            iota(parent.begin(), parent.end(), 0);
        }
        
        disjoint_set_union(int length) {
            parent.resize(length + 1);
            iota(parent.begin(), parent.end(), 0);
            count.resize(length + 1, 1);
        }
        
        int find(int node) {
            return parent[node] = parent[node] == node ? node : find(parent[node]);
        }
        
        void joint(int x, int y) {
            int rootx = find(x), rooty = find(y);

            if (rootx != rooty) {
                parent[rooty] = rootx;
                count[rootx] += count[rooty];
            }
        }
};
\end{lstlisting}