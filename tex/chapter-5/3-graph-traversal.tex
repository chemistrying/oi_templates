\subsection{Graph Traversal}
\subsubsection{Dijkstra's Algorithm}
A single source weighted shortest path algorithm in $O(n\log n)$
\begin{lstlisting}
void dijkstra(vector<int>& visited, int node){
    min_heap<pair<int, int>> pq;
    pq.push({0, node});
    visited[node] = 0;

    while (!pq.empty()){
        pair<int, int> curr = pq.top();
        pq.pop();

        if (curr.first > visited[curr.second]) continue;
        
        for (auto a : graph[curr.second]){
            if (curr.first + a.second < visited[a.first]){
                visited[a.first] = curr.first + a.second;
                pq.push({curr.first + a.second, a.first});
            }
        }
    }
}
\end{lstlisting}
\subsubsection{Bellman-Ford Algorithm}
A single source weighted shortest path algorithm which supports negative edges in $O(VE)$.
\begin{lstlisting}
vector<long long> dist(n, inf * inf);
dist[root] = 0;

for (int t = 0; t < n - 1; t++) {
    for (int i = 0; i < n; i++) {
        for (pair<int, int> other : graph[i]) {
            int u = i, v = other.first;
            if (dist[u] != inf * inf && dist[v] > dist[u] + other.second) {
                dist[v] = dist[u] + other.second;
            }
        }
    }
}

// for checking negative cycles
for (int i = 0; i < n; i++) {
    for (pair<int, int> other : graph[i]) {
        int u = i, v = other.first;
        if (dist[u] != inf * inf && dist[v] > dist[u] + other.second) {
            cout << "NEGATIVE CYCLE\n";
            return 0;
        }
    }
}
\end{lstlisting}
\subsubsection{Floyd-Warshall Algorithm}
An all pairs weighted shortest path algorithm which supports negative edges in $O(V^3)$.
\begin{lstlisting}
vector dist(n, vector(n, inf + inf));
for (int i = 0; i < n; i++) {
    dist[i][i] = 0;
}

for (int i = 0; i < m; i++) {
    long long u, v, w;
    cin >> u >> v >> w;
    dist[u][v] = w;
}

for (int k = 0; k < n; k++) {
    for (int i = 0; i < n; i++) {
        for (int j = 0; j < n; j++) {
            if (dist[i][k] != inf + inf && dist[k][j] != inf + inf) {
                if (dist[i][k] + dist[k][j] < dist[i][j]) {
                    dist[i][j] = dist[i][k] + dist[k][j];
                }
            }
        }
    }
}

// for checking negative cycles
for (int i = 0; i < n; i++) {
    for (int j = 0; j < n; j++) {
        for (int k = 0; k < n; k++) {
            if (dist[i][k] + dist[k][j] + dist[j][i] < 0) {
                cout << "NEGATIVE CYCLE\n";
                return 0;
            }
        }
    }
}
\end{lstlisting}
\subsubsection{0-1 BFS}
If the weights of the edges are either 0 or 1, we do not have to use a priority queue. 
Instead, we can use a deque. 
When we do insertion, if the newly added edge is 0, place it in front of the deque; otherwise, place it at the back of the deque.
\begin{lstlisting}
vector<int> dist(n, INF);
dist[s] = 0;
deque<int> dq;
dq.push_front(s);
while (!dq.empty()) {
    int node = dq.front();
    dq.pop_front();
    for (auto edge : graph[node]) {
        int other = edge.first;
        int weight = edge.second;
        if (dist[other] + w < dist[node]) {
            dust[other] = dist[node] + w;
            if (w == 1)
                q.push_back(other);
            else
                q.push_front(other);
        }
    }
}
\end{lstlisting}